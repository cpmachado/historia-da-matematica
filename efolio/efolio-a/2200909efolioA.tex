\documentclass[11pt,a4paper]{article}
\usepackage[utf8]{inputenc}
\usepackage[portuguese]{babel}
\usepackage[T1]{fontenc}
\usepackage[ddmmyyyy]{datetime}
\usepackage[margin=2cm]{geometry}
\usepackage{amsfonts}
\usepackage{amsmath}
\usepackage{amssymb}
\usepackage{amsthm}
\usepackage{mathtools}
\usepackage{csvsimple}
\usepackage{tcolorbox}
\usepackage{csquotes}
\usepackage{subcaption}
\usepackage{graphicx}
\usepackage{titling}
\usepackage{tikz}
\usepackage[style=apa]{biblatex}
\usepackage{hyperref}

\usetikzlibrary {calc,shapes.geometric}

\graphicspath{{imagem}}
\addbibresource{bibliografia.bib}

% Dados
\author{Carlos Augusto Gonçalves Collaço e Pinto Machado}
\newcommand{\numeroEstudante}{2200909}
\newcommand{\UC}{História da Matemática}
\newcommand{\codigoUC}{21166}
\newcommand{\docentes}{António Araújo}
\newcommand{\curso}{Licenciatura em Matemática e Aplicações}
\newcommand{\turma}{1}
\newcommand{\anoLectivo}{2023-2024}
\date{27/11/2023}

\title{\numeroEstudante - \UC | \codigoUC - \theauthor}

\pagenumbering{arabic}

\hypersetup{
	pdftitle={\thetitle},
	pdfsubject={\thetitle},
	pdfauthor={\theauthor},
	pdfkeywords={\thetitle}
}

\makeindex

\newcommand{\exercicio}{
	\addtocounter{section}{1}
	\section*{Problema \arabic{section}.}
	\addcontentsline{toc}{section}{Problema \arabic{section}.}
}

\begin{document}

\input{capa/default.tex}

\clearpage

\exercicio{}

\paragraph{\underline{Derivação da fórmula explícita dos números pentagonais}}\hfill

\vspace{0.25cm}

O primeiro número pentagonal é $p_1 = 1$. A cada iteração da construção acrescenta-se ao pentágono
de ordem $n$, 3 novos lados de $n+1$ pontos, pelo que cada lado exterior perfaz $n+1$ pontos. Dado
que estes 3 novos lados partilham 2 pontos, obtemos um incremento de
$3(n + 1) - 2 = 3n + 1$ pontos, ao pentágono de ordem $n$.
Pelo que obtemos a seguinte sequência:
\begin{align*}
	\csvreader[
		head to column names,
		late after line = \\
	]{data/ex1.csv} {}{
		p_\k &= \sk = \pk
	}
	\ldots&
\end{align*}

\noindent
De modo a obter uma fórmula para calcular o k-ésimo número pentagonal,comecemos por observar que
temos uma progressão aritmética de razão $3k -2$, pelo que obtemos a seguinte definição por
recorrência:
\begin{align*}
	p_k =
	\begin{cases}
		1, &\text{ se }k = 1\\
		3k - 2 + p_{k - 1}, &\text{ se }k > 1\\
	\end{cases}
\end{align*}

\noindent
Dado que se $k=1$, tem-se $3k - 2 = 1$, e que:
\begin{align*}
	p_k
		= \sum_{i=1}^{k} (3i - 2)
		= 3S_k - 2k
		= 3 \frac{k(k + 1)}{2} -2k
		= \frac{3k(k + 1) - 4k}{2}
		= \frac{k(3k - 1)}{2}
\end{align*}
Pelo que temos uma fórmula explícita para o k-ésimo termo dos números
pentagonais, definida por:
\begin{equation}\label{eq:k-numero-pentagonal}
	p_k = \sum_{i=1}^{k} (3i-2) = \frac{k(3k - 1)}{2}, \quad k \in \mathbb{N}
\end{equation}

\noindent
Esta fórmula está presente em \cite[pág.31]{Kline1990Vol1}, sob a forma $\frac{3n^2 -n }{2}$.

\paragraph{\underline{JUSTIFICAÇÃO GRÁFICA NA PRÓXIMA PÁGINA}}

\paragraph{\underline{Cálculo de $p_{100}$}}\hfill

\vspace{0.25cm}

\noindent
Com recurso à fórmula descrita na equação
\textbf{(\ref{eq:k-numero-pentagonal})}, tem-se que:
\begin{align*}
	p_{100}
	&= \frac{100(3 \cdot 100 - 1)}{2} = 50 \cdot (300 - 1)
	= 15000 - 50 = 14950
\end{align*}

\clearpage

\paragraph{\underline{Justificação gráfica}}\hfill

\vspace{0.25cm}

A seguinte construção é baseada à fornecida
em \cite[problema 16.]{HMAF1},
e em \cite[pág. 33]{Kline1990Vol1}.\\
Para construir este diagrama, segue-se os seguintes passos:
\begin{enumerate}
	\item Começa-se com o pentágono de ordem 1, que é o ponto no topo.
	\item Constrói-se o pentágono de ordem 2, ao adicionar os 4 pontos, cada com o ponto do topo
		desenham uma recta sobre
		o qual os restantes vértices do pentágono de iterações posteriores vão assentar.
	\item Para construir o pentágono de ordem $n+1$ extende-se os vértices do pentágono por um
		ponto, e conectam-se essas arestas, que terão $n+1$ pontos de comprimento.
	\item Dado que os pontos que conectam essas faces são comuns, temos mais $3(n + 1) - 2 = 3n + 1$
		pontos no pentágono de ordem $n+1$, do que no de ordem $n$.
\end{enumerate}

\begin{figure}[h]
	\centering
	\begin{tikzpicture}[scale=0.75, transform shape]
	\pgfmathsetmacro{\angle}{360/5}
	\coordinate (p1) at (0, 0);
	\filldraw (p1) circle (2pt);
	
	\foreach \i in {1,...,4} {
		\pgfmathsetmacro{\x}{cos(\i*\angle + 90)}
		\pgfmathsetmacro{\y}{sin(\i*\angle + 90) - 1}
		\coordinate (p2\i) at (\x, \y);
		\filldraw (p2\i) circle (2pt);
		\draw (p2\i) -- (p1);
	}
	\draw 
		(p21) --
		($(p21)!0.5!(p22)$) node[below left] {$2$} --
		(p22) --
		($(p22)!0.5!(p23)$) node[below] {$2$} --
		(p23) --
		($(p23)!0.5!(p24)$) node[below right] {$2$} --
		(p24);

	\foreach \i in {1,...,4} {
		\pgfmathsetmacro{\x}{2*cos(\i*\angle + 90)}
		\pgfmathsetmacro{\y}{2*sin(\i*\angle + 90) - 2}
		\coordinate (pn\i) at (\x, \y);
		\filldraw (pn\i) circle (2pt);
		\draw[dashed] (p2\i) -- (pn\i);
	}

	\draw (pn1) -- (pn2) -- (pn3) -- (pn4);

	\node[below left] at ($(pn1)!0.5!(pn2)$) {$n$};
	\node[below] at ($(pn2)!0.5!(pn3)$) {$n$};
	\node[below right] at ($(pn3)!0.5!(pn4)$) {$n$};

	\foreach \i in {1,...,4} {
		\pgfmathsetmacro{\x}{3*cos(\i*\angle + 90)}
		\pgfmathsetmacro{\y}{3*sin(\i*\angle + 90) - 3}
		\coordinate (pnp\i) at (\x, \y);
		\filldraw (pnp\i) circle (2pt);
		\draw (pn\i) -- (pnp\i);
	}

	\draw
		(pnp1) --
		($(pnp1)!0.5!(pnp2)$) node[below left] {$n + 1$} --
		(pnp2) --
		($(pnp2)!0.5!(pnp3)$) node[below] {$n + 1$} --
		(pnp3) --
		($(pnp3)!0.5!(pnp4)$) node[below right] {$n + 1$} --
		(pnp4);

	\node[above left] at (p1) {$p_1$};
	\node[above left] at (p21) {$p_2$};
	\node[above left] at (pn1) {$p_n$};
	\node[above left] at (pnp1) {$p_{n + 1}$};

\end{tikzpicture}

	\caption{Construção iterativa de números pentagonais}
\end{figure}





\clearpage
\exercicio{}

O método geométrico de corte e colagem aparece no manual\parencite[pág. 77-78]{Estrada2000} como uma
hipótese de Neugebauer sobre qual os babilónios aproximaram o valor de $\sqrt{2}$.
Este método pode ser utilizado para iterativamente calcular a raíz de qualquer $N \in \mathbb{N}$.

\vspace{0.25cm}

\noindent
Consideremos:
\begin{itemize}
	\item $N\in \mathbb{N}$, o número pelo qual pretendemos calcular a raíz quadrada;
	\item $x_0 \in \mathbb{N}$ é uma estimativa por excesso;
	\item $y_0 \in \mathbb{N}$ é uma estimativa por defeito;
	\item $N = x_0 y_0$
\end{itemize}

\noindent
Do manual\parencite[pág. 77]{Estrada2000}, tem-se as seguintes relações de recorrência, por
Neugebauer:

\begin{align*}
	\begin{cases}
		x_{n + 1} = \frac{x_n + y_n}{2}\\
		y_{n + 1} = \frac{N}{x_{n+1}}
	\end{cases}
	\implies N = x_n \cdot y_n
\end{align*}

\noindent
De modo a criar um critério de paragem eficaz, comecemos por provar quer $y_n \leq x_n \leq N,
\forall n \in \mathbb{N}$.

\begin{enumerate}
	\item Caso base: para $n = 0$, temos por definição que $x_0 \geq y_0$.
	\item Supomos que: $x_n \geq y_n$(Hipótese de Indução)
	\item Pretendemos provar que: $x_{n+1} \geq y_{n+1}$ (Tese de Indução)
	\item Verifiquemos se $x_{n + 1} \geq y_{n + 1}$, com base na Hipótese de Indução:
		\begin{align*}
			x_{n + 1} = \frac{x_n + y_n}{2} &\geq \frac{N}{x_{n + 1}} = y_{n +1} \\ 
			\iff x_n^2 + y_nx_n &\geq 2x_ny_n\\
			\iff x_n^2 &\geq y_nx_n\\
			\iff x_n &\geq y_n
		\end{align*}
		Damos deste modo o passo de indução $x_n > y_n$ de indução, podemos concluir que $x_{n +1}
		\geq y_{n + 1}$.
\end{enumerate}

Os digramas da figura na página seguinte descreve o processo de iteração geométrica pelo método de corte e
colagem. Deste diagrama também derivamos a fórmula de erro.

A partir do diagrama (c) temos que a fórmula de erro para uma dada iteração $n$, porque
$x_n \geq y_n$,  é majorado por:

\begin{align*}
	\varepsilon_n = x_n - y_n
\end{align*}


\clearpage
\begin{figure}[ht!]
	\centering
	\begin{subfigure}[b]{0.3\textwidth}
	\centering
	\input{tikz/rectangulo-0.tex}
	\caption{}
	\end{subfigure}
	\begin{subfigure}[b]{0.30\textwidth}
	\centering
	\begin{tikzpicture}[scale=0.75]
	\draw (0, 0) rectangle (5, 3);
	\draw (3, 0) -- (3, 3);
	\draw[dashed] (4, 0) -- (4, 3);
	\node[left] at (-0.25, 1.5) {$y_n$};
	\node[above] at (2.5, 3.25) {$x_n$};
	\draw[thick,<->] (0, 3.25) -- (5, 3.25);
	\draw[thick,<->] (-0.25, 0) -- (-0.25, 3);
\end{tikzpicture}

	\caption{}
	\end{subfigure}
	\begin{subfigure}[b]{0.30\textwidth}
	\centering
	\begin{tikzpicture}[scale=0.75]
	\draw (0, 0) rectangle (4, 3);
	\draw (0, 3) rectangle (3, 4);
	\draw (3, 0) -- (3, 3);
	\node[left] at (-0.25, 1.5) {$y_n$};
	\node[above] at (2, 4.25) {$x_{n + 1}$};
	\node[left] at (6.25, 2) {$x_{n + 1}$};
	\draw[thick,<->] (0, 4.25) -- (4, 4.25);
	\draw[thick,<->] (-0.25, 0) -- (-0.25, 3);
	\draw[thick,<->] (4.25, 0) -- (4.25, 4);
\end{tikzpicture}

	\caption{}
	\end{subfigure}

	\bigskip

	\begin{subfigure}{0.4\textwidth}
	\centering
	\input{tikz/rectangulo-0_3.tex}
	\caption{}
	\end{subfigure}
	\begin{subfigure}{0.4\textwidth}
	\centering
	\begin{tikzpicture}[scale=0.75]
	\draw (0, 0) rectangle (4, 15/4);
	\draw (15/4, 0) -- (15/4, 15/4);
	\node[left] at (-0.25, 15/8) {$y_{n + 1}$};
	\node[above] at (2, 4) {$x_{n + 1}$};
	\draw[thick,<->] (0, 4) -- (4, 4);
	\draw[thick,<->] (-0.25, 0) -- (-0.25, 15/4);
\end{tikzpicture}

	\caption{}
	\end{subfigure}
	\caption{Processo iterativo de cálculo de aproximações de $\sqrt{N}$}%
\end{figure}


\paragraph{\underline{Cálculo $\sqrt{15}$, com $\varepsilon \leq 10^{-5}$}}
\hfill

\paragraph{Resposta} $\sqrt{15} = \frac{7440}{1921} \pm \frac{1}{962816}$

\begin{table}[h]
	\centering
	\label{tab:raiz-15-iteracoes}
	\begin{tabular}{|r|l|l|l|}
		\hline
		$i$ & $x_i$ & $y_i$ & $\varepsilon$\\
		\hline
		\csvreader[
			no head,
			late after line = \\\hline
		]{data/ex2.csv} {}{
			\csvcoli & \csvcolii &\csvcoliii &\csvcoliv
		}
	\end{tabular}
	\caption{Iterações do algoritmo}
\end{table}

\begin{align*}
	N
	&= 15 = 5 \cdot 3\\
	&\begin{cases}
		x_0 &= 5\\
		y_0 &= 3\\
		e_0 &= 2
	\end{cases}\\
	&\begin{cases}
		x_1 &= \frac{x_0 + y_0}{2} = 4\\
		y_1 &= \frac{2N}{x_0 + y_0} = \frac{15}{4}\\
		e_1 &= \frac{1}{4}
	\end{cases}\\
	&\begin{cases}
		x_2 &= \frac{x_1 + y_1}{2} = \frac{31}{8}\\
		y_2 &= \frac{2N}{x_1 + y_1} = \frac{120}{31}\\
		e_2 &= \frac{1}{248}
	\end{cases}\\
	&\begin{cases}
		x_3 &= \frac{x_2 + y_2}{2} = \frac{1921}{496}\\
		y_3 &= \frac{2N}{x_2 + y_2} = \frac{7440}{1921}\\
		e_2 &= \frac{1}{952816}
	\end{cases}
\end{align*}



\clearpage

\printbibliography[heading=bibintoc,title={Bibliografia}]

\end{document}
