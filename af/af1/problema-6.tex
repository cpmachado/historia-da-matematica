\problema{6}

\paragraph{} Pretende-se calcular o complemento de $\frac{2}{3}+\frac{1}{15}$
à unidade, com o método dos escribas egípcios. Recorreu-se à análise do
problema 22 do papiro de Rhind\parencite[pág. 37-38]{Estrada2000}, que define
o seguinte método:

\begin{enumerate}
	\item Calcular o m.m.c. dos denominadores das fracções;
	\item Resolver a equação linear a partir dos numeradores das fracções
		normalizadas ao m.m.c.;
	\item Calcular a expansão unitária do resultado a dividir pelo m.m.c.,
		caso este não seja uma fracção unitária.
\end{enumerate}

\begin{align*}
	\text{m.m.c.}(3, 15) &= 15 \\
	\frac{2}{3} + \frac{1}{15} + \frac{a}{15} &= 1\\
	\iff10 + 1 + a &= 15 \\
		\iff a &= 4
\end{align*}

\begin{table}[h]
	\centering
	\begin{tabular}{cr|cr}
		&1&&15\\
		&$\overline{\overline{3}}$&&10\\
		&$\overline{2}$&&$7 \; \overline{2}$\\
		&$\overline{3}$&&5\\
		&$\overline{4}$&&$3 \; \overline{2} \; \overline{4}$\\
		\textbackslash&$\overline{5}$&\textbackslash&3\\
		\textbackslash&$\overline{15}$&\textbackslash&1\\
		\hline
		&$5 \; \overline{15}$&&4
	\end{tabular}
\end{table}

\paragraph{} O complemento procurado pode ser expresso por
$5 \; \overline{15}$.

