\exercicio{}

\paragraph{\underline{Derivação da fórmula explícita dos números pentagonais}}\hfill

\vspace{0.25cm}

O primeiro número pentagonal é $p_1 = 1$. A cada iteração da construção acrescenta-se ao pentágono
de ordem $n$, 3 novos lados de $n+1$ pontos, pelo que cada lado exterior perfaz $n+1$ pontos. Dado
que estes 3 novos lados partilham 2 pontos, obtemos um incremento de
$3(n + 1) - 2 = 3n + 1$ pontos, ao pentágono de ordem $n$.
Pelo que obtemos a seguinte sequência:
\begin{align*}
	\csvreader[
		head to column names,
		late after line = \\
	]{data/ex1.csv} {}{
		p_\k &= \sk = \pk
	}
	\ldots&
\end{align*}

\noindent
De modo a obter uma fórmula para calcular o k-ésimo número pentagonal,comecemos por observar que
temos uma progressão aritmética de razão $3k -2$, pelo que obtemos a seguinte definição por
recorrência:
\begin{align*}
	p_k =
	\begin{cases}
		1, &\text{ se }k = 1\\
		3k - 2 + p_{k - 1}, &\text{ se }k > 1\\
	\end{cases}
\end{align*}

\noindent
Dado que se $k=1$, tem-se $3k - 2 = 1$, e que:
\begin{align*}
	p_k
		= \sum_{i=1}^{k} (3i - 2)
		= 3S_k - 2k
		= 3 \frac{k(k + 1)}{2} -2k
		= \frac{3k(k + 1) - 4k}{2}
		= \frac{k(3k - 1)}{2}
\end{align*}
Pelo que temos uma fórmula explícita para o k-ésimo termo dos números
pentagonais, definida por:
\begin{equation}\label{eq:k-numero-pentagonal}
	p_k = \sum_{i=1}^{k} (3i-2) = \frac{k(3k - 1)}{2}, \quad k \in \mathbb{N}
\end{equation}

\noindent
Esta fórmula está presente em \cite[pág.31]{Kline1990Vol1}, sob a forma $\frac{3n^2 -n }{2}$.

\paragraph{\underline{JUSTIFICAÇÃO GRÁFICA NA PRÓXIMA PÁGINA}}

\paragraph{\underline{Cálculo de $p_{100}$}}\hfill

\vspace{0.25cm}

\noindent
Com recurso à fórmula descrita na equação
\textbf{(\ref{eq:k-numero-pentagonal})}, tem-se que:
\begin{align*}
	p_{100}
	&= \frac{100(3 \cdot 100 - 1)}{2} = 50 \cdot (300 - 1)
	= 15000 - 50 = 14950
\end{align*}

\clearpage

\paragraph{\underline{Justificação gráfica}}\hfill

\vspace{0.25cm}

A seguinte construção é baseada à fornecida
em \cite[problema 16.]{HMAF1},
e em \cite[pág. 33]{Kline1990Vol1}.\\
Para construir este diagrama, segue-se os seguintes passos:
\begin{enumerate}
	\item Começa-se com o pentágono de ordem 1, que é o ponto no topo.
	\item Constrói-se o pentágono de ordem 2, ao adicionar os 4 pontos, cada com o ponto do topo
		desenham uma recta sobre
		o qual os restantes vértices do pentágono de iterações posteriores vão assentar.
	\item Para construir o pentágono de ordem $n+1$ extende-se os vértices do pentágono por um
		ponto, e conectam-se essas arestas, que terão $n+1$ pontos de comprimento.
	\item Dado que os pontos que conectam essas faces são comuns, temos mais $3(n + 1) - 2 = 3n + 1$
		pontos no pentágono de ordem $n+1$, do que no de ordem $n$.
\end{enumerate}

\begin{figure}[h]
	\centering
	\begin{tikzpicture}[scale=0.75, transform shape]
	\pgfmathsetmacro{\angle}{360/5}
	\coordinate (p1) at (0, 0);
	\filldraw (p1) circle (2pt);
	
	\foreach \i in {1,...,4} {
		\pgfmathsetmacro{\x}{cos(\i*\angle + 90)}
		\pgfmathsetmacro{\y}{sin(\i*\angle + 90) - 1}
		\coordinate (p2\i) at (\x, \y);
		\filldraw (p2\i) circle (2pt);
		\draw (p2\i) -- (p1);
	}
	\draw 
		(p21) --
		($(p21)!0.5!(p22)$) node[below left] {$2$} --
		(p22) --
		($(p22)!0.5!(p23)$) node[below] {$2$} --
		(p23) --
		($(p23)!0.5!(p24)$) node[below right] {$2$} --
		(p24);

	\foreach \i in {1,...,4} {
		\pgfmathsetmacro{\x}{2*cos(\i*\angle + 90)}
		\pgfmathsetmacro{\y}{2*sin(\i*\angle + 90) - 2}
		\coordinate (pn\i) at (\x, \y);
		\filldraw (pn\i) circle (2pt);
		\draw[dashed] (p2\i) -- (pn\i);
	}

	\draw (pn1) -- (pn2) -- (pn3) -- (pn4);

	\node[below left] at ($(pn1)!0.5!(pn2)$) {$n$};
	\node[below] at ($(pn2)!0.5!(pn3)$) {$n$};
	\node[below right] at ($(pn3)!0.5!(pn4)$) {$n$};

	\foreach \i in {1,...,4} {
		\pgfmathsetmacro{\x}{3*cos(\i*\angle + 90)}
		\pgfmathsetmacro{\y}{3*sin(\i*\angle + 90) - 3}
		\coordinate (pnp\i) at (\x, \y);
		\filldraw (pnp\i) circle (2pt);
		\draw (pn\i) -- (pnp\i);
	}

	\draw
		(pnp1) --
		($(pnp1)!0.5!(pnp2)$) node[below left] {$n + 1$} --
		(pnp2) --
		($(pnp2)!0.5!(pnp3)$) node[below] {$n + 1$} --
		(pnp3) --
		($(pnp3)!0.5!(pnp4)$) node[below right] {$n + 1$} --
		(pnp4);

	\node[above left] at (p1) {$p_1$};
	\node[above left] at (p21) {$p_2$};
	\node[above left] at (pn1) {$p_n$};
	\node[above left] at (pnp1) {$p_{n + 1}$};

\end{tikzpicture}

	\caption{Construção iterativa de números pentagonais}
\end{figure}




