\problema{4}

\paragraph{} Usa-se como referência o Problema 25 do Papiro de
Rhind\parencite[pág.29]{Estrada2000}.

\vspace{0.5cm}

\alinea{i}

\begin{table}[h]
	\centering
	\begin{tabular}{cr|cr}
		&1&&6\\
		\textbackslash&2&\textbackslash&12\\
		\textbackslash&3&\textbackslash&24\\
		\hline
		&5&&30
	\end{tabular}
\end{table}

\begin{align*}
	&\text{Quociente: } 5 \\
	&\text{Resto: } 32 - 30 = 2
\end{align*}

\alinea{ii}

\begin{table}[h]
	\centering
	\begin{tabular}{cr|cr}
		&1&&6\\
		\textbackslash&2&\textbackslash&12\\
		\textbackslash&3&\textbackslash&24\\
		&$\overline{\overline{3}}$&&4\\
		&$\overline{2}$&&3\\
		\textbackslash&$\overline{3}$&\textbackslash&2\\
		\hline
		&$5 \; \overline{3}$&&32
	\end{tabular}
\end{table}

\begin{align*}
	\text{Quociente: } 5 \; \overline{3}
\end{align*}

